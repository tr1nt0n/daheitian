\documentclass[12pt]{article}
\usepackage{fontspec}
\usepackage[utf8]{inputenc}
\setmainfont{Bodoni 72 Book}
\usepackage[paperwidth=11in,paperheight=17in,margin=1in,headheight=0.0in,footskip=0.5in,includehead,includefoot,portrait]{geometry}
\usepackage[absolute]{textpos}
\TPGrid[0.5in, 0.25in]{23}{24}
\parindent=0pt
\parskip=12pt
\usepackage{nopageno}
\usepackage{graphicx}
\graphicspath{ {./images/} }
\usepackage{amsmath}
\usepackage{tikz}
\newcommand*\circled[1]{\tikz[baseline=(char.base)]{
            \node[shape=circle,draw,inner sep=1pt] (char) {#1};}}

\begin{document}

\vspace*{2\baselineskip}

\begingroup
\begin{center}
\huge KRÄFTE
\end{center}
\endgroup

\vspace*{2\baselineskip}

\begingroup
\begin{center}
Die für dieses Stück erforderlichen \textbf{Mindest}kräfte sind so:
\end{center}

\hspace{67mm} \textbf{\circled{1} Zwei} Flöten \\

\hspace{67mm} \textbf{\circled{2} Zwei} Oboen, spielen Oboe \textbf{und} Cor Anglais \\

\hspace{67mm} \textbf{\circled{3} Zwei} Bassklarinetten \\

\hspace{67mm} \textbf{\circled{4} Zwei} Fagotte \\

\hspace{67mm} \textbf{\circled{5} Zwei} Hörner in F \\

\hspace{67mm} \textbf{\circled{6} Zwei} Trompeten in C \\

\hspace{67mm} \textbf{\circled{7} Zwei} Tenorposaunen \\

\hspace{67mm} \textbf{\circled{8} Zwei} Tuben \\

\hspace{67mm} \textbf{\circled{9} Ein} Klavier \\

\hspace{67mm} \textbf{\circled{10} Eine} Harfe \\

\hspace{67mm} \textbf{\circled{11} Ein} Pauker, spielen zwei Pauken, eine tiefe und eine hohe. \\

\hspace{67mm} \textbf{\circled{12} Zwei} Schlagzeuger \\

\hspace{67mm} \textbf{\circled{13} Zwölf} erste Geigen \\

\hspace{67mm} \textbf{\circled{14} Zwölf} zweite Geigen \\

\hspace{67mm} \textbf{\circled{15} Zwölf} Bratschen \\

\hspace{67mm} \textbf{\circled{16} Zwölf} Violoncelli \\

\hspace{67mm} \textbf{\circled{17} Sechs} Kontrabässe \\

\endgroup


\pagebreak

\begingroup
\begin{center}
\huge HINWEISE FÜR DIE INTERPRETEN
\end{center}
\endgroup

\vspace*{2\baselineskip}

\begingroup
\textbf{Allgemein: \circled{1} Vorzeichen} werden für jeden Takt gesetzt, aber sie werden nochmal gesetzt, wenn die gleiche Note später im selben Takt auftritt - außer die Note wird unmittelbar wiederholt. \textbf{\circled{2} Dynamik}, gefolgt von einem Pluszeichen, bedeutet, dass zwischen der notierten Dynamik und der nächsten Standarddynamikstufe gespielt werden soll. So zeigt \textbf{pp +} an, dass zwischen Pianissimo und Piano gespielt werden soll. \textbf{\circled{3}} Wenn \textbf{Dynamiken mit Rhythmen innerhalb einer Dauer verknüpft werden}, wird anstelle einer traditionellen Verbindung \textbf{eine gerade, durchgezogene Linie} über die Länge der Dauer verwendet. \textbf{\circled{4} Instrumentaltechniken} gelten nur für die Note, mit der sie verbunden sind. Wenn eine Technik länger als eine Note bestehen muss, umspannt eine \textbf{Hakenlinie} die Musik, in der die Technik aktiv ist. \textbf{\circled{5} Pfeile} kennzeichnen einen allmählichen Wechsel von einer Technik oder einem Tempo zu einer anderen. \textbf{\circled{6} Vorschlagsnoten vor} einer Note sollten direkt vor dem Rhythmus gespielt werden, Vorschlagsnoten \textbf{nach} einer Note sollten ganz am Ende der Dauer der betreffenden Note gespielt werden. \textbf{\circled{7} Wenn eine ganze Orchestergruppe eine frei interpretierte Technik spielt}, wie z. B. das grafische Vibrato in den Holzbläsern bei Takt 230, \\ 
\begin{center}
\includegraphics[scale=0.45]{vibrato.png}
\end{center}
\endgroup

\begingroup
oder die annähernden Glissandi in den Streichern ab Takt 53 \\
\begin{center}
\includegraphics[scale=0.55]{string_glissandi.png}
\end{center}
\endgroup

\begingroup
oder ein accelerando / ritardando wie so, \\
\begin{center}
\includegraphics[scale=0.20]{ritardando.png}
\end{center}
\endgroup

\begingroup
muss \textbf{nicht die gesamte Orchestergruppe genau unisono interpretieren.} Vielmehr ist eine Variation des freien Parameters von Individuum zu Individuum erwünscht. \textbf{\circled{8} Fermaten} und ihre Längen sind wie folgt zu interpretieren:
\endgroup

\begingroup
\hspace{67mm} \circled{1} Sehr kurz \hspace{5mm} \includegraphics[scale=0.05]{ganz_kurz.png}
\endgroup

\begingroup
\hspace{67mm} \circled{2} Kurz \hspace{13mm} \includegraphics[scale=0.045]{kurz.png}
\endgroup

\begingroup
\hspace{67mm} \circled{3} Mittel \hspace{10mm} \includegraphics[scale=0.04]{mittel.png}
\endgroup

\begingroup
\hspace{67mm} \circled{4} Lang \hspace{12mm} \includegraphics[scale=0.035]{lang.png}
\endgroup

\begingroup
\hspace{67mm} \circled{5} Sehr lang \hspace{4.5mm} \includegraphics[scale=0.045]{sehr_lang.png}
\endgroup

\pagebreak

\begingroup
\textbf{\circled{9}} Ein nach oben oder unten gerichteter \textbf{dreieckiger Notenkopf} zeigt an, dass die höchst- oder tiefstmögliche Tonhöhe gespielt werden soll. \\
\textbf{\circled{10}} Die in diesem Stück verwendeten \textbf{gleichschwebenden Intervalle} sind \textbf{Halbtöne, Vierteltöne} und \textbf{Achteltöne.} Ihre Symbole lauten wie folgt:
\endgroup

\begingroup
\hspace{67mm} \circled{1} Ein Viertelton höher \hspace{5mm} \includegraphics[scale=0.06]{qtone_sharp.png}
\endgroup

\begingroup
\hspace{67mm} \circled{2} Ein Viertelton tiefer \hspace{5mm} \includegraphics[scale=0.07]{qtone_flat.png}
\endgroup 

\begingroup
\circled{3} Jedes Vorzeichen kann mit einem Pfeil oben oder unten verändert werden, was bedeutet, dass die Tonhöhe um einen Achtelton erniedrigt erhöht oder erniedrigt wird. \\
\begin{center}
\includegraphics[scale=0.07]{eighthtone.png}
\end{center}
\endgroup

\begingroup
\textbf{\circled{11} Rationale Intervalle} werden durch die Verwendung des \textbf{Helmholtz-Ellis-Vorzeichensystems} in Kombination mit \textbf{Cent-Abweichungen von der gleichschwebenden Stimmung} für die Verwendung mit einem elektronischen Stimmgerät angegeben. Wenn keine Beispieltonhöhe mit der Cent-Abweichung angegeben ist, ist die Markierung eine Abweichung vom \textbf{nächstgelegenen ,,Standard" Vorzeiche.} In Ermangelung elektronischer Stimmgeräte sind Näherungswerte für diese Abweichungen zulässig. Wenn die Helmholtz-Ellis-Notation nicht angegeben ist, sind die Tonhöhen wie üblich zu spielen. \textbf{\circled{12}}  Im Allgemeinen bedeutet ein mehrstimmiges Notensystem ein traditionelles \textbf{Divisi}. Eine Ausnahme bilden Passagen wie Takt 150 in den tiefe Rohrblattinstrumente. \\
\begin{center}
\includegraphics[scale=0.30]{compound_rhythm.png}
\end{center}
Hier weist der nach oben gerichtete Balken auf ein zusammengesetzter Rhythmus hin, einen effektiven Akzent, und sollte nicht als Divisi interpretiert werden. \textbf{\circled{13} Wenn eine Passage für bestimmte Mitglieder desselben Orchestergruppe gilt}, wird die folgende Syntax verwendet: \textbf{,,1."} bedeutet, dass nur das erste Mitglied der Gruppe spielen soll. \textbf{,,1.|2."} bedeutet, dass nur das erste und zweite Mitglied der Gruppe spielen soll. \textbf{,,2.:"} gibt an, dass alle Mitglieder außer dem ersten spielen sollen.
\endgroup

\begingroup
\textbf{Flöten: \circled{1} Kleine stiellose Noten, die mit einem gebrochenen Balken beginnen}, wie hier, \\
\begin{center}
\includegraphics[scale=0.50]{whistletones.png}
\end{center}
zeigen \textbf{Whistletones} an. \textbf{\circled{2} Die Vorschlagsnoten auf dem Taktschlag} ab Takt 246 \\
\begin{center}
\includegraphics[scale=0.20]{flute_overblowing.png}
\end{center}
zeigen das \textbf{Überblasen durch die Teiltöne eines Grundtons} an, hier in Klammern. Diese Geste sollte schnell und explosiv, aber dennoch schön gespielt werden, wobei der Grundton anschließend für den Rest der eingeklammerten Noten gehalten wird. \textbf{\circled{3} Die Rhythmen dieser beiden Techniken} können relativ frei interpretiert werden und müssen daher nicht als Unisono zwischen allen Flötisten:innen interpretiert werden.
\endgroup

\pagebreak

\begingroup
\textbf{Rohrblattinstrumente: \circled{1} Mehrklänge} werden mit Griffdiagrammen oberhalb des Grundtons angezeigt. \textbf{\circled{2} Rhythmisierte Klangfarbenänderungen} werden als eingekreiste Zahl über einer Note notiert ( z. B. \circled{1}, \circled{2} oder \circled{3} ), wobei höhere Zahlen eine größere Abweichung in Klangfarbe und Tonhöhe bedeuten.
\endgroup

\begingroup
\textbf{Oboen: \circled{1}} Zwischen den Takten 114 und 115 findet ein sehr schneller Wechsel zwischen Cor Anglais und Oboe statt. Es wird empfohlen, dass der zweite Oboist zu Beginn von Takt 113 oder 114 aufhört zu spielen, um den Wechsel zu Takt 115 rechtzeitig zu vollziehen, und dass der erste Oboist später in 115 hinzustößt, nachdem er die Musik in 114 auf dem Cor Anglais beendet hat.
\endgroup

\begingroup
\textbf{Bassklarinetten: \circled{1} Diese Partitur ist so transponiert}, dass die notierte Tonhöhe \textbf{eine große None} über der klingenden Tonhöhe liegt.
\endgroup

\begingroup
\textbf{Fagotte: \circled{1} Pizzicato} wird mit gekreuzten Notenköpfen notiert, und \textbf{schmetternde Klänge} werden mit einem Pfeil auf dem Notenhals in Richtung des Notenkopfes notiert, wie in Allgemein unter Punkt 12 dargestellt. Die Anleitungen für diese Techniken sind so:
\begin{center}
,,Das \textbf{Pizzicato} entsteht beim Fagott durch sehr kurzes Schnalzen der Rohrspitze auf den Lippen . . . Notwendig ist nur eine kurze, trockene Lippenbewegung auf der Rohrspitze; auf den Blasdruck des Zwerchfells wird verzichtet. Die Bewegung gleicht der Aussprache des Buchstabens >>P<<." \\

- Pascal Gallois, \textit{Die Spieltechnik des Fagotts} Seite 47 Abs. 1
\end{center}
\endgroup

\begingroup
\begin{center}
,,Der \textbf{schmetternde Klang} ähnelt dem Pizzicato, ist jedoch immer sehr kräftig und erfordert den Blasdruck des Zwerchfells. Er entsteht durch ein sehr hartes und energisches Schnalzen der Rohrspitze auf den Lippen und erinnert an den schmetternden Klang einer Posaune . . ." \\

- Pascal Gallois, \textit{Die Spieltechnik des Fagotts} Seite 47 Abs. 3
\end{center}
\endgroup

\begingroup
\textbf{Klavier: \circled{1} Kreuzartikulationen} bedeuten, dass die betreffenden Saiten im Inneren des Klaviers mit dem linken Arm gedämpft werden, während man die Tasten spielt.
\endgroup

\begingroup
\textbf{Harfe: \circled{1}} Mit Ausnahme des Es, das bei Takt 176 benötigt wird, ist die Harfe immer auf \textbf{A, B, C, Dis, E, Fis} und \textbf{G} gestimmt. \textbf{\circled{2}} Der Harfenspieler:in sollte mit einer \textbf{Plastikkarte} und einer \textbf{kleinen Stimmgabel in G} ausgestattet sein.
\endgroup

\begingroup
\textbf{Schlagzeug: \circled{1} Die Instrumente} und ihre \textbf{Werkzeug} des \textbf{ersten} Schlagzeuger sind so: \\
a.) Ein kleiner ( hohes ) \textbf{Triangel,} und ein \textbf{Triangel Schlager} \\
b.) Ein \textbf{Bangu \setmainfont{Source Han Serif SC Bold}\selectfont{ ( 板鼓 ), } }und zwei kleine \textbf{Bambusstäbchen} ( Diese können bei Bedarf durch \textbf{Trommelstöcke} ersetzt werden, obwohl dies nicht bevorzugt wird ) \\
c.) Ein Satz \textbf{Röhrenglocken,} und zwei \textbf{Röhrenglockenschlägel} \\
d.) Ein große \textbf{Tanggu \setmainfont{Source Han Serif SC Bold}\selectfont{ ( 堂鼓 ), } }und zwei \textbf{Trommelstöcke}\\
\textbf{\circled{3} Die Instrumente} und ihre \textbf{Werkzeug} des \textbf{zweiter} Schlagzeuger sind so: \\
a.) Ein \textbf{Glockenspiel}, und zwei kleine \textbf{Plastikschläger} \\
b.) Ein kleiner \textbf{Amboss}, und zwei kleine \textbf{Hämmerchen} \\
c.) Ein kleiner \textbf{Gong der chinesische Oper\setmainfont{Source Han Serif SC Bold}\selectfont{ ( 小鑼 ) } }  \\
d.) Ein mittelgroßer \textbf{Gong der chinesische Oper \setmainfont{Source Han Serif SC Bold}\selectfont{ ( 中型鑼 ), } } und zwei kleine, harte \textbf{Gongschlägel}\\
e.) Ein großer \textbf{Tam-Tam} ( vorbereitet mit \textbf{Ketten} an der Vorderseite ), und ein \textbf{Bogen} \\
f.) Ein kleiner ( hohes ) \textbf{Triangel,} und ein \textbf{Triangel Schlager} \\
\endgroup

\begingroup
\textbf{Streicher: \circled{1}} Die in diesem Beitrag verwendeten \textbf{Abkürzungen} sind so: \\
\circled{1} \textbf{DP} steht für \textbf{dietro ponticello}. Das bedeutet, dass die Saiten zwischen dem Steg und der Umspinnung zu spielen sind. \\
\circled{2} \textbf{Steg} steht für \textbf{direkt auf dem Steg}. Bei dieser Spieltechnik sollten alle Saiten gedämpft werden, um einen tonlosen Klang zu erzeugen, es sei denn, es ist eine Tonhöhe mit gekreuztem Notenkopf angegeben; in diesem Fall sollte diese Tonhöhe gegriffen werden. \\
\circled{3} \textbf{MSP} steht für \textbf{molto sul ponticello}. Bei dieser Technik sollte die Hälfte der Bogenhaare direkt auf dem Steg und die andere Hälfte auf den Saiten liegen. \\
\circled{4} \textbf{SP} steht für \textbf{sul ponticello}. \\
\circled{5} \textbf{Ord.} steht für \textbf{ordinario}. \\
\circled{6} \textbf{ST} steht für \textbf{sul tasto}. \\
\circled{7} \textbf{MST} steht für \textbf{molto sul tasto}. Bei dieser Technik sollte der Bogen so nah wie möglich an der Mitte des Griffbretts sein. \\
\textbf{\circled{2} Rautenförmige Notenköpfe|} zeigen an, dass man die Tonhöhe mit Druck berühren soll, als ob man einen Flageolett-Ton spielt, egal ob ein Flageolett erklingt oder nicht. \textbf{Weiße rautenförmige Notenköpfe auf einem normalen Notenkopf} weisen auf künstlichen Flageolett hin. \textbf{\circled{3}} Wenn ein \textbf{Trillando} mit einem \textbf{Glissando} gepaart ist, sollte sich das Intervall dieses Trillandos ( immer ein Halbton ) mit der Hauptnote bewegen. \textbf{\circled{3} Ein vierzeiliges Notensystem} zeigt an, dass auf offen Saiten gespielt werden soll, wobei die \textbf{oberste Zeile} die \textbf{erste Saite}, die \textbf{nächste Zeile} die \textbf{zweite Saite} und so weiter anzeigt. \textbf{\circled{4} Eine geschwungene Doppelpfeil-Artikulation}, wie unten, \\ 
\begin{center}
\includegraphics[scale=0.20]{twist_bow.png}
\end{center}
zeigt an, dass der Bogen auf die Saite au'talon gesetzt und gedreht werden soll, molto gridato.
\endgroup

\end{document}