\documentclass[12pt]{article}
\usepackage{fontspec}
\usepackage[utf8]{inputenc}
\setmainfont{Bodoni 72 Book}
\usepackage[paperwidth=11in,paperheight=17in,margin=1in,headheight=0.0in,footskip=0.5in,includehead,includefoot,portrait]{geometry}
\usepackage[absolute]{textpos}
\TPGrid[0.5in, 0.25in]{23}{24}
\parindent=0pt
\parskip=12pt
\usepackage{nopageno}
\usepackage{graphicx}
\graphicspath{ {./images/} }
\usepackage{amsmath}
\usepackage{tikz}
\newcommand*\circled[1]{\tikz[baseline=(char.base)]{
            \node[shape=circle,draw,inner sep=1pt] (char) {#1};}}

\begin{document}

\vspace*{2\baselineskip}

\begingroup
\begin{center}
\huge KRÄFTE
\end{center}
\endgroup

\vspace*{2\baselineskip}

\begingroup
\begin{center}
Die für dieses Stück erforderlichen \textbf{Mindest}kräfte sind so:
\end{center}

\hspace{67mm} \textbf{\circled{1} Zwei} Flöten \\

\hspace{67mm} \textbf{\circled{2} Zwei} Oboen \\

\hspace{67mm} \textbf{\circled{3} Zwei} Bassklarinetten \\

\hspace{67mm} \textbf{\circled{4} Zwei} Fagotte \\

\hspace{67mm} \textbf{\circled{5} Zwei} Hörner in F \\

\hspace{67mm} \textbf{\circled{6} Zwei} Trompeten in C \\

\hspace{67mm} \textbf{\circled{7} Zwei} Tenorposaunen \\

\hspace{67mm} \textbf{\circled{8} Zwei} Tuben \\

\hspace{67mm} \textbf{\circled{9} Ein} Klavier \\

\hspace{67mm} \textbf{\circled{10} Eine} Harfe \\

\hspace{67mm} \textbf{\circled{11} Ein} Pauker, spielt zwei Pauken, eine tiefe und eine hohe. \\

\hspace{67mm} \textbf{\circled{12} Zwei} Schlagzeuger \\

\hspace{67mm} \textbf{\circled{13} Acht} erste Geigen \\

\hspace{67mm} \textbf{\circled{14} Acht} zweite Geigen \\

\hspace{67mm} \textbf{\circled{15} Acht} Bratschen \\

\hspace{67mm} \textbf{\circled{16} Acht} Violoncelli \\

\hspace{67mm} \textbf{\circled{17} Vier} Kontrabässe \\

\endgroup


\pagebreak

\begingroup
\begin{center}
\huge HINWEISE FÜR DIE INTERPRETEN
\end{center}
\endgroup

\vspace*{2\baselineskip}

\begingroup
\textbf{Schlagzeug: \circled{1} Die Instrumente} des \textbf{ersten} Schlagzeuger sind so: \\
a.) Ein kleiner ( hohes ) \textbf{Triangel } \\
b.) Ein \textbf{Bangu \setmainfont{Source Han Serif SC Bold}\selectfont{ ( 板鼓 ) } } \\
c.) Ein Satz \textbf{Röhrenglocken } \\
b.) Ein große \textbf{Tanggu \setmainfont{Source Han Serif SC Bold}\selectfont{ ( 堂鼓 ) } } \\
\textbf{\circled{2} Die Werkzeuge} des \textbf{ersten} Schlagzeuger sind so: \\
a.) Ein \textbf{Triangel Schlager} \\
b.) Zwei kleine \textbf{Bambusstäbchen} ( Diese können bei Bedarf durch \textbf{Trommelstöcke} ersetzt werden, obwohl dies nicht bevorzugt wird ) \\
c.) Zwei \textbf{Röhrenglockenschlägel} \\
d.) Zwei \textbf{\textbf{Trommelstöcke} } \\
\textbf{\circled{3} Die Instrumente} des \textbf{zweiter} Schlagzeuger sind so: \\
a.) Ein \textbf{Glockenspiel} \\
b.) Ein kleiner \textbf{Amboss} \\
c.) Ein kleiner \textbf{Gong der chinesisch Oper\setmainfont{Source Han Serif SC Bold}\selectfont{ ( 小锣 ) } }  \\
d.) Ein mittelgroßer \textbf{Gong der chinesisch Oper \setmainfont{Source Han Serif SC Bold}\selectfont{ ( 中型锣 ) } } \\
e.) Ein großer \textbf{Tam-Tam} ( vorbereitet mit \textbf{Ketten} an der Vorderseite ) \\
\textbf{\circled{4} Die Werkzeuge} des \textbf{zweiter} Schlagzeuger sind so: \\
a.) Zwei kleine \textbf{Plastikschläger} \\
b.) Zwei kleine \textbf{Hämmerchen} \\
c.) Zwei kleine, harte \textbf{Gongschlägel} \\
d.) Ein \textbf{Bogen} \\
\endgroup

\end{document}