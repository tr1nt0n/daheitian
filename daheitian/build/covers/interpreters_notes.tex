\documentclass[12pt]{article}
\usepackage{fontspec}
\usepackage[utf8]{inputenc}
\setmainfont{Bodoni 72 Book}
\usepackage[paperwidth=11in,paperheight=17in,margin=1in,headheight=0.0in,footskip=0.5in,includehead,includefoot,portrait]{geometry}
\usepackage[absolute]{textpos}
\TPGrid[0.5in, 0.25in]{23}{24}
\parindent=0pt
\parskip=12pt
\usepackage{nopageno}
\usepackage{graphicx}
\graphicspath{ {./images/} }
\usepackage{amsmath}
\usepackage{tikz}
\newcommand*\circled[1]{\tikz[baseline=(char.base)]{
            \node[shape=circle,draw,inner sep=1pt] (char) {#1};}}

\begin{document}

\vspace*{2\baselineskip}

\begingroup
\begin{center}
\huge KRÄFTE
\end{center}
\endgroup

\vspace*{2\baselineskip}

\begingroup
\begin{center}
Die für dieses Stück erforderlichen \textbf{Mindest}kräfte sind so:
\end{center}

\hspace{67mm} \textbf{\circled{1} Zwei} Flöten \\

\hspace{67mm} \textbf{\circled{2} Zwei} Oboen, spielen Oboe \textbf{und} Cor Anglais \\

\hspace{67mm} \textbf{\circled{3} Zwei} Bassklarinetten \\

\hspace{67mm} \textbf{\circled{4} Zwei} Fagotte \\

\hspace{67mm} \textbf{\circled{5} Zwei} Hörner in F \\

\hspace{67mm} \textbf{\circled{6} Zwei} Trompeten in C \\

\hspace{67mm} \textbf{\circled{7} Zwei} Tenorposaunen \\

\hspace{67mm} \textbf{\circled{8} Zwei} Tuben \\

\hspace{67mm} \textbf{\circled{9} Ein} Klavier \\

\hspace{67mm} \textbf{\circled{10} Eine} Harfe \\

\hspace{67mm} \textbf{\circled{11} Ein} Pauker, spielen zwei Pauken, eine tiefe und eine hohe. \\

\hspace{67mm} \textbf{\circled{12} Zwei} Schlagzeuger \\

\hspace{67mm} \textbf{\circled{13} Acht} erste Geigen \\

\hspace{67mm} \textbf{\circled{14} Acht} zweite Geigen \\

\hspace{67mm} \textbf{\circled{15} Acht} Bratschen \\

\hspace{67mm} \textbf{\circled{16} Acht} Violoncelli \\

\hspace{67mm} \textbf{\circled{17} Vier} Kontrabässe \\

\endgroup


\pagebreak

\begingroup
\begin{center}
\huge HINWEISE FÜR DIE INTERPRETEN
\end{center}
\endgroup

\vspace*{2\baselineskip}

\begingroup
\textbf{Allgemein: \circled{1} Vorzeichen} werden für jeden Takt gesetzt, aber sie werden nochmal gesetzt, wenn die gleiche Note später im selben Takt auftritt - außer die Note wird unmittelbar wiederholt. \textbf{\circled{2} Instrumentaltechniken} gelten nur für die Note, mit der sie verbunden sind. Wenn eine Technik länger als eine Note bestehen muss, umspannt eine \textbf{Hakenlinie} die Musik, in der die Technik aktiv ist. \textbf{\circled{3} Pfeile} kennzeichnen einen allmählichen Wechsel von einer Technik oder einem Tempo zu einer anderen. \textbf{\circled{4} Vorschlagsnoten vor} einer Note sollten direkt vor dem Rhythmus gespielt werden, Vorschlagsnoten \textbf{nach} einer Note sollten ganz am Ende der Dauer der betreffenden Note gespielt werden. \textbf{\circled{5} Wenn ein ganzer Orchesterteil eine frei interpretierte Technik spielt}, wie z. B. das grafische Vibrato in den Holzbläsern bei Takt 230, \\ 
\begin{center}
\includegraphics[scale=0.45]{vibrato.png}
\end{center}
\endgroup

\begingroup
oder die annähernden Glissandi in den Streichern ab Takt 53 \\
\begin{center}
\includegraphics[scale=0.55]{string_glissandi.png}
\end{center}
\endgroup

\begingroup
oder ein accelerando / ritardando wie so, \\
\begin{center}
\includegraphics[scale=0.20]{ritardando.png}
\end{center}
\endgroup

\begingroup
muss \textbf{nicht der gesamte Orchesterteil genau unisono interpretieren.} Vielmehr ist eine Variation des freien Parameters von Individuum zu Individuum erwünscht. \textbf{\circled{6} Fermaten} und ihre Längen sind wie folgt zu interpretieren:
\endgroup

\begingroup
\hspace{67mm} \circled{1} Sehr kurz \hspace{5mm} \includegraphics[scale=0.05]{ganz_kurz.png}
\endgroup

\begingroup
\hspace{67mm} \circled{2} Kurz \hspace{13mm} \includegraphics[scale=0.045]{kurz.png}
\endgroup

\begingroup
\hspace{67mm} \circled{3} Mittel \hspace{10mm} \includegraphics[scale=0.04]{mittel.png}
\endgroup

\begingroup
\hspace{67mm} \circled{4} Lang \hspace{12mm} \includegraphics[scale=0.035]{lang.png}
\endgroup

\begingroup
\hspace{67mm} \circled{5} Sehr lang. \hspace{2.75mm} \includegraphics[scale=0.045]{sehr_lang.png}
\endgroup

\pagebreak

\begingroup
\circled{7} Die in diesem Stück verwendeten \textbf{gleichschwebenden Intervalle} sind \textbf{Halbtöne, Vierteltöne} und \textbf{Achteltöne.} Ihre Symbole lauten wie folgt:
\endgroup

\begingroup
\hspace{67mm} \circled{1} Ein Viertelton höher \hspace{5mm} \includegraphics[scale=0.06]{qtone_sharp.png}
\endgroup

\begingroup
\hspace{67mm} \circled{2} Ein Viertelton tiefer \hspace{5mm} \includegraphics[scale=0.07]{qtone_flat.png}
\endgroup 

\begingroup
\circled{3} Jedes Vorzeichen kann mit einem Pfeil oben oder unten verändert werden, was bedeutet, dass die Tonhöhe um einen Achtelton erhöht wird. \\
\begin{center}
\includegraphics[scale=0.07]{eighthtone.png}
\end{center}
\endgroup

\begingroup
\textbf{\circled{8} Rationale Intervalle} werden durch die Verwendung des \textbf{Helmholtz-Ellis-Vorzeichensystems} in Kombination mit \textbf{Cent-Abweichungen von der gleichschwebenden Stimmung} für die Verwendung mit einem elektronischen Stimmgerät angegeben. Wenn keine Beispieltonhöhe mit der Cent-Abweichung angegeben ist, ist die Markierung eine Abweichung vom \textbf{nächstgelegenen ,,Standard" Vorzeiche.} In Ermangelung elektronischer Stimmgeräte sind Näherungswerte für diese Abweichungen zulässig. Wenn die Helmholtz-Ellis-Notation nicht angegeben ist, sind die Tonhöhen wie üblich zu spielen.
\endgroup

\begingroup
\textbf{Schlagzeug: \circled{1} Die Instrumente} des \textbf{ersten} Schlagzeuger sind so: \\
a.) Ein kleiner ( hohes ) \textbf{Triangel } \\
b.) Ein \textbf{Bangu \setmainfont{Source Han Serif SC Bold}\selectfont{ ( 板鼓 ) } } \\
c.) Ein Satz \textbf{Röhrenglocken } \\
b.) Ein große \textbf{Tanggu \setmainfont{Source Han Serif SC Bold}\selectfont{ ( 堂鼓 ) } } \\
\textbf{\circled{2} Die Werkzeuge} des \textbf{ersten} Schlagzeuger sind so: \\
a.) Ein \textbf{Triangel Schlager} \\
b.) Zwei kleine \textbf{Bambusstäbchen} ( Diese können bei Bedarf durch \textbf{Trommelstöcke} ersetzt werden, obwohl dies nicht bevorzugt wird ) \\
c.) Zwei \textbf{Röhrenglockenschlägel} \\
d.) Zwei \textbf{\textbf{Trommelstöcke} } \\
\textbf{\circled{3} Die Instrumente} des \textbf{zweiter} Schlagzeuger sind so: \\
a.) Ein \textbf{Glockenspiel} \\
b.) Ein kleiner \textbf{Amboss} \\
c.) Ein kleiner \textbf{Gong der chinesische Oper\setmainfont{Source Han Serif SC Bold}\selectfont{ ( 小鑼 ) } }  \\
d.) Ein mittelgroßer \textbf{Gong der chinesische Oper \setmainfont{Source Han Serif SC Bold}\selectfont{ ( 中型鑼 ) } } \\
e.) Ein großer \textbf{Tam-Tam} ( vorbereitet mit \textbf{Ketten} an der Vorderseite ) \\
\textbf{\circled{4} Die Werkzeuge} des \textbf{zweiter} Schlagzeuger sind so: \\
a.) Zwei kleine \textbf{Plastikschläger} \\
b.) Zwei kleine \textbf{Hämmerchen} \\
c.) Zwei kleine, harte \textbf{Gongschlägel} \\
d.) Ein \textbf{Bogen} \\
\endgroup

\end{document}