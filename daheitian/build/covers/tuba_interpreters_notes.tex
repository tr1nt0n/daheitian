\documentclass[12pt]{article}
\usepackage{fontspec}
\usepackage[utf8]{inputenc}
\setmainfont{Bodoni 72 Book}
\usepackage[paperwidth=9in,paperheight=12in,margin=1in,headheight=0.0in,footskip=0.5in,includehead,includefoot,portrait]{geometry}
\usepackage[absolute]{textpos}
\TPGrid[0.5in, 0.25in]{23}{24}
\parindent=0pt
\parskip=12pt
\usepackage{nopageno}
\usepackage{graphicx}
\graphicspath{ {./images/} }
\usepackage{amsmath}
\usepackage{tikz}
\newcommand*\circled[1]{\tikz[baseline=(char.base)]{
            \node[shape=circle,draw,inner sep=1pt] (char) {#1};}}

\begin{document}

\begingroup
\begin{center}
\huge HINWEISE FÜR DIE INTERPRETEN
\end{center}
\endgroup

\vspace*{2\baselineskip}


\begingroup
\textbf{Allgemein: \circled{1} Vorzeichen} werden für jeden Takt gesetzt, aber sie werden nochmal gesetzt, wenn die gleiche Note später im selben Takt auftritt - außer die Note wird unmittelbar wiederholt. \textbf{\circled{2} Dynamik}, gefolgt von einem Pluszeichen, bedeutet, dass zwischen der notierten Dynamik und der nächsten Standarddynamikstufe gespielt werden soll. So zeigt \textbf{pp +} an, dass zwischen Pianissimo und Piano gespielt werden soll. \textbf{\circled{3} Flache Glissandi} werden in ähnlicher Weise wie Bindebögen verwendet, aber während Bindebögen auf die Darstellung metrischer Pulsgruppierungen während einer einzelnen Note beschränkt sind, binden flache Glissandi komponierte Rhythmen, um als \textbf{Ankernoten für dynamische Veränderungen} innerhalb einer anhaltenden einzelnen Note verwendet zu werden. Die Interpreten müssen sich nicht darum kümmern, ob ein solches flaches Glissando ein ,,echtes Glissando" eines Halbtons ist, da ein solches ,,echtes Glissando" \textbf{immer auch mit Vorzeichen} angezeigt wird. \textbf{\circled{4} Instrumentaltechniken} gelten nur für die Note, mit der sie verbunden sind. Wenn eine Technik länger als eine Note bestehen muss, umspannt eine \textbf{Hakenlinie} die Musik, in der die Technik aktiv ist. \textbf{\circled{5} Pfeile} kennzeichnen einen allmählichen Wechsel von einer Technik oder einem Tempo zu einer anderen. \textbf{\circled{6} Vorschlagsnoten vor} einer Note sollten direkt vor dem Rhythmus gespielt werden, Vorschlagsnoten \textbf{nach} einer Note sollten ganz am Ende der Dauer der betreffenden Note gespielt werden. \textbf{\circled{7} Wenn eine ganze Orchestergruppe eine frei interpretierte Technik spielt}, müssen \textbf{nicht die gesamte Orchestergruppe genau unisono interpretieren.} Vielmehr ist eine Variation der freien Parameters von Individuum zu Individuum erwünscht.  \\

\textbf{\circled{8} Fermaten} und ihre Längen sind wie folgt zu interpretieren:
\endgroup

\begingroup
\hspace{67mm} \circled{1} Sehr kurz \hspace{5mm} \includegraphics[scale=0.05]{ganz_kurz.png}
\endgroup

\begingroup
\hspace{67mm} \circled{2} Kurz \hspace{13mm} \includegraphics[scale=0.045]{kurz.png}
\endgroup

\begingroup
\hspace{67mm} \circled{3} Mittel \hspace{10mm} \includegraphics[scale=0.04]{mittel.png}
\endgroup

\begingroup
\hspace{67mm} \circled{4} Lang \hspace{12mm} \includegraphics[scale=0.035]{lang.png}
\endgroup

\begingroup
\hspace{67mm} \circled{5} Sehr lang \hspace{4.5mm} \includegraphics[scale=0.045]{sehr_lang.png}
\endgroup

\begingroup
\textbf{\circled{9}} Da diese Parameter von Instrument zu Instrument und von Lautstärke zu Lautstärke variieren können, wird die \textbf{höchst- bzw. tiefstmögliche Tonhöhe} eines Instruments, die nicht auf eine bestimmte Harmonie, sondern auf einen \textbf{Effekt} abzielt, mit einem \textbf{nach oben bzw. nach unten gerichteten dreieckigen Notenkopf} angezeigt.\\
\textbf{\circled{10}} ( \textit{Dieser Punkt ist für dieses Instrument nicht relevant, wurde aber aus Gründen der numerischen Konsistenz beibehalten.} ) 
\endgroup

\begingroup
\textbf{\circled{11} Eine X/X-Taktart} mit gestrichelten Taktstrichen und Sekundenmarkierungen über dem Notensystem zeigt \textbf{ametrische Musik} an, bei der ein Takt \textbf{eine Sekunde} dauert. Um die Synchronisierung zu erleichtern, werden etwa alle vier Sekunden \textbf{,,Meilensteine"} in Form von Pfeilen über dem Notensystem angegeben. \textbf{\circled{12}}  Im Allgemeinen bedeutet ein mehrstimmiges Notensystem ein traditionelles \textbf{Divisi}. \textbf{\circled{13} Wenn eine Passage für bestimmte Mitglieder desselben Orchestergruppe gilt}, wird die folgende Syntax verwendet: \textbf{,,1. soli"} bedeutet, dass nur das erste Mitglied der Gruppe spielen soll. \textbf{,,1.|2. soli"} bedeutet, dass nur das erste und zweite Mitglied der Gruppe spielen soll. \textbf{\circled{14} Einsätze} werden gegeben, wenn die Musiker nach einer langen Pause, die keine Grand Pause Fermate ist, zu spielen beginnen müssen. Diese Einsätze sind immer mit ,,\textbf{Einsatz:}" gekennzeichnet, gefolgt von der \textbf{Bezeichnung des Instruments}, von dem die Einsatz stammt. Die Schriftgröße der Einsätze ist \textbf{deutlich kleiner} als die Schriftgröße der übrigen Stimme und wird immer mit dem Hinweis ,,\textbf{Ende des Einsatzes}" abgeschlossen.
\endgroup

\begingroup
\textbf{Tuba: \circled{1} Wenn keine erste Tuba zur Verfügung steht}, kann die Stimme von einer gedämpften \textbf{Bassposaune} übernommen werden.
\endgroup

\end{document}