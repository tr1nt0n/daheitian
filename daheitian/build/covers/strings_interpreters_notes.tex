\documentclass[12pt]{article}
\usepackage{fontspec}
\usepackage[utf8]{inputenc}
\setmainfont{Bodoni 72 Book}
\usepackage[paperwidth=9in,paperheight=12in,margin=1in,headheight=0.0in,footskip=0.5in,includehead,includefoot,portrait]{geometry}
\usepackage[absolute]{textpos}
\TPGrid[0.5in, 0.25in]{23}{24}
\parindent=0pt
\parskip=12pt
\usepackage{nopageno}
\usepackage{graphicx}
\graphicspath{ {./images/} }
\usepackage{amsmath}
\usepackage{tikz}
\newcommand*\circled[1]{\tikz[baseline=(char.base)]{
            \node[shape=circle,draw,inner sep=1pt] (char) {#1};}}

\begin{document}

\begingroup
\begin{center}
\huge HINWEISE FÜR DIE INTERPRETEN
\end{center}
\endgroup

\vspace*{2\baselineskip}

\begingroup
\textbf{Allgemein: \circled{1} Vorzeichen} werden für jeden Takt gesetzt, aber sie werden nochmal gesetzt, wenn die gleiche Note später im selben Takt auftritt - außer die Note wird unmittelbar wiederholt. \textbf{\circled{2} Dynamik}, gefolgt von einem Pluszeichen, bedeutet, dass zwischen der notierten Dynamik und der nächsten Standarddynamikstufe gespielt werden soll. So zeigt \textbf{pp +} an, dass zwischen Pianissimo und Piano gespielt werden soll. \textbf{\circled{3}} Wenn \textbf{Dynamiken mit Rhythmen innerhalb einer Dauer verknüpft werden}, wird anstelle einer traditionellen Verbindung \textbf{eine gerade, durchgezogene Linie} über die Länge der Dauer verwendet. \textbf{\circled{4} Instrumentaltechniken} gelten nur für die Note, mit der sie verbunden sind. Wenn eine Technik länger als eine Note bestehen muss, umspannt eine \textbf{Hakenlinie} die Musik, in der die Technik aktiv ist. \textbf{\circled{5} Pfeile} kennzeichnen einen allmählichen Wechsel von einer Technik oder einem Tempo zu einer anderen. \textbf{\circled{6} Vorschlagsnoten vor} einer Note sollten direkt vor dem Rhythmus gespielt werden, Vorschlagsnoten \textbf{nach} einer Note sollten ganz am Ende der Dauer der betreffenden Note gespielt werden. \textbf{\circled{7} Wenn eine ganze Orchestergruppe eine frei interpretierte Technik spielt}, wie z.B. die annähernden Glissandi in den Streichern ab Takt 53 \\
\begin{center}
\includegraphics[scale=0.35]{string_glissandi.png}
\end{center}
\endgroup

\begingroup
oder ein accelerando / ritardando wie so, \\
\begin{center}
\includegraphics[scale=0.20]{ritardando.png}
\end{center}
\endgroup

\begingroup
muss \textbf{nicht die gesamte Orchestergruppe genau unisono interpretieren.} Vielmehr ist eine Variation der freien Parameters von Individuum zu Individuum erwünscht.

\pagebreak

\textbf{\circled{8} Fermaten} und ihre Längen sind wie folgt zu interpretieren:
\endgroup

\begingroup
\hspace{67mm} \circled{1} Sehr kurz \hspace{5mm} \includegraphics[scale=0.05]{ganz_kurz.png}
\endgroup

\begingroup
\hspace{67mm} \circled{2} Kurz \hspace{13mm} \includegraphics[scale=0.045]{kurz.png}
\endgroup

\begingroup
\hspace{67mm} \circled{3} Mittel \hspace{10mm} \includegraphics[scale=0.04]{mittel.png}
\endgroup

\begingroup
\hspace{67mm} \circled{4} Lang \hspace{12mm} \includegraphics[scale=0.035]{lang.png}
\endgroup

\begingroup
\hspace{67mm} \circled{5} Sehr lang \hspace{4.5mm} \includegraphics[scale=0.045]{sehr_lang.png}
\endgroup

\begingroup
\textbf{\circled{9}} Ein nach oben oder unten gerichteter \textbf{dreieckiger Notenkopf} zeigt an, dass die höchst- oder tiefstmögliche Tonhöhe gespielt werden soll. \\
\textbf{\circled{10}} Die in diesem Stück verwendeten \textbf{gleichschwebenden Intervalle} sind \textbf{Halbtöne}, und \textbf{Vierteltöne}. Ihre Symbole lauten wie folgt:
\endgroup

\begingroup
\hspace{67mm} \circled{1} Ein Viertelton höher \hspace{5mm} \includegraphics[scale=0.06]{qtone_sharp.png}
\endgroup

\begingroup
\hspace{67mm} \circled{2} Ein Viertelton tiefer \hspace{5mm} \includegraphics[scale=0.07]{qtone_flat.png}
\endgroup 

\begingroup
\textbf{\circled{11} Rationale Intervalle} werden durch die Verwendung des \textbf{Helmholtz-Ellis-Vorzeichensystems} in Kombination mit \textbf{Cent-Abweichungen von der gleichschwebenden Stimmung} für die Verwendung mit einem elektronischen Stimmgerät angegeben. Wenn keine Beispieltonhöhe mit der Cent-Abweichung angegeben ist, ist die Markierung eine Abweichung vom \textbf{nächstgelegenen ,,Standard" Vorzeiche.} In Ermangelung elektronischer Stimmgeräte sind Näherungswerte für diese Abweichungen zulässig. Wenn die Helmholtz-Ellis-Notation nicht angegeben ist, sind die Tonhöhen wie üblich zu spielen. \textbf{\circled{12}}  Im Allgemeinen bedeutet ein mehrstimmiges Notensystem ein traditionelles \textbf{Divisi}. \textbf{\circled{13} Wenn eine Passage für bestimmte Mitglieder desselben Orchestergruppe gilt}, wird die folgende Syntax verwendet: \textbf{,,1."} bedeutet, dass nur das erste Mitglied der Gruppe spielen soll. \textbf{,,1.|2."} bedeutet, dass nur das erste und zweite Mitglied der Gruppe spielen soll. \textbf{,,2.:"} gibt an, dass alle Mitglieder außer dem ersten spielen sollen.
\endgroup

\pagebreak

\begingroup
\textbf{Streicher: \circled{1}} Die in dieser Partitur verwendeten \textbf{Abkürzungen} sind so: \\
\circled{1} \textbf{DP} steht für \textbf{dietro ponticello}. Das bedeutet, dass die Saiten zwischen dem Steg und der Umspinnung zu spielen sind. \\
\circled{2} \textbf{Steg} steht für \textbf{direkt auf dem Steg}. Bei dieser Spieltechnik sollten alle Saiten gedämpft werden, um einen tonlosen Klang zu erzeugen, es sei denn, es ist eine Tonhöhe mit gekreuztem Notenkopf angegeben; in diesem Fall sollte diese Tonhöhe gegriffen werden. \\
\circled{3} \textbf{MSP} steht für \textbf{molto sul ponticello}. Bei dieser Technik sollte die Hälfte der Bogenhaare direkt auf dem Steg und die andere Hälfte auf den Saiten liegen. \\
\circled{4} \textbf{SP} steht für \textbf{sul ponticello}. \\
\circled{5} \textbf{Ord.} steht für \textbf{ordinario}. \\
\circled{6} \textbf{ST} steht für \textbf{sul tasto}. \\
\circled{7} \textbf{MST} steht für \textbf{molto sul tasto}. Bei dieser Technik sollte der Bogen so nah wie möglich an der Mitte des Griffbretts sein. \\
\textbf{\circled{2} Rautenförmige Notenköpfel} zeigen an, dass man die Tonhöhe mit Druck berühren soll, als ob man einen Flageolett-Ton spielt, egal ob ein Flageolett erklingt oder nicht. \textbf{Weiße rautenförmige Notenköpfe auf einem normalen Notenkopf} weisen auf künstlichen Flageolett hin. \\ \textbf{\circled{3}} Wenn ein \textbf{Trille} mit einem \textbf{Glissando} gepaart ist, sollte sich das Intervall dieses Trillandos ( immer ein Halbton ) mit der Hauptnote bewegen. \\ \textbf{\circled{4} Ein vierzeiliges Notensystem} zeigt an, dass auf offen Saiten gespielt werden soll, wobei die \textbf{oberste Zeile} die \textbf{erste Saite}, die \textbf{nächste Zeile} die \textbf{zweite Saite} und so weiter anzeigt. \textbf{\circled{5} Eine geschwungene Doppelpfeil-Artikulation}, wie unten, \\ 
\begin{center}
\includegraphics[scale=0.12]{twist_bow.png}
\end{center}
zeigt an, dass der Bogen auf die Saite au'talon gesetzt und gedreht werden soll, molto gridato.
\endgroup

\end{document}