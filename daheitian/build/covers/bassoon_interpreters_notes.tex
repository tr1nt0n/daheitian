\documentclass[12pt]{article}
\usepackage{fontspec}
\usepackage[utf8]{inputenc}
\setmainfont{Bodoni 72 Book}
\usepackage[paperwidth=9in,paperheight=12in,margin=1in,headheight=0.0in,footskip=0.5in,includehead,includefoot,portrait]{geometry}
\usepackage[absolute]{textpos}
\TPGrid[0.5in, 0.25in]{23}{24}
\parindent=0pt
\parskip=12pt
\usepackage{nopageno}
\usepackage{graphicx}
\graphicspath{ {./images/} }
\usepackage{amsmath}
\usepackage{tikz}
\newcommand*\circled[1]{\tikz[baseline=(char.base)]{
            \node[shape=circle,draw,inner sep=1pt] (char) {#1};}}

\begin{document}

\begingroup
\begin{center}
\huge HINWEISE FÜR DIE INTERPRETEN
\end{center}
\endgroup

\vspace*{2\baselineskip}

\begingroup
\textbf{Allgemein: \circled{1} Vorzeichen} werden für jeden Takt gesetzt, aber sie werden nochmal gesetzt, wenn die gleiche Note später im selben Takt auftritt - außer die Note wird unmittelbar wiederholt. \textbf{\circled{2} Dynamik}, gefolgt von einem Pluszeichen, bedeutet, dass zwischen der notierten Dynamik und der nächsten Standarddynamikstufe gespielt werden soll. So zeigt \textbf{pp +} an, dass zwischen Pianissimo und Piano gespielt werden soll. \textbf{\circled{3}} Wenn \textbf{Dynamiken mit Rhythmen innerhalb einer Dauer verknüpft werden}, wird anstelle einer traditionellen Verbindung \textbf{eine gerade, durchgezogene Linie} über die Länge der Dauer verwendet. \textbf{\circled{4} Instrumentaltechniken} gelten nur für die Note, mit der sie verbunden sind. Wenn eine Technik länger als eine Note bestehen muss, umspannt eine \textbf{Hakenlinie} die Musik, in der die Technik aktiv ist. \textbf{\circled{5} Pfeile} kennzeichnen einen allmählichen Wechsel von einer Technik oder einem Tempo zu einer anderen. \textbf{\circled{6} Vorschlagsnoten vor} einer Note sollten direkt vor dem Rhythmus gespielt werden, Vorschlagsnoten \textbf{nach} einer Note sollten ganz am Ende der Dauer der betreffenden Note gespielt werden. \textbf{\circled{7} Wenn eine ganze Orchestergruppe eine frei interpretierte Technik spielt}, wie z. B. das grafische Vibrato in den Holzbläsern bei Takt 230, \\ 
\begin{center}
\includegraphics[scale=0.30]{vibrato.png}
\end{center}
\endgroup

\begingroup
muss \textbf{nicht die gesamte Orchestergruppe genau unisono interpretieren.} Vielmehr ist eine Variation des freien Parameters von Individuum zu Individuum erwünscht. \textbf{\circled{8} Fermaten} und ihre Längen sind wie folgt zu interpretieren:
\endgroup

\begingroup
\hspace{67mm} \circled{1} Sehr kurz \hspace{5mm} \includegraphics[scale=0.05]{ganz_kurz.png}
\endgroup

\begingroup
\hspace{67mm} \circled{2} Kurz \hspace{13mm} \includegraphics[scale=0.045]{kurz.png}
\endgroup

\begingroup
\hspace{67mm} \circled{3} Mittel \hspace{10mm} \includegraphics[scale=0.04]{mittel.png}
\endgroup

\begingroup
\hspace{67mm} \circled{4} Lang \hspace{12mm} \includegraphics[scale=0.035]{lang.png}
\endgroup

\begingroup
\hspace{67mm} \circled{5} Sehr lang \hspace{4.5mm} \includegraphics[scale=0.045]{sehr_lang.png}
\endgroup

\pagebreak

\begingroup
\textbf{\circled{9}}  Im Allgemeinen bedeutet ein mehrstimmiges Notensystem ein traditionelles \textbf{Divisi}. Eine Ausnahme bilden Passagen wie Takt 150 in den tiefe Rohrblattinstrumente. \\
\begin{center}
\includegraphics[scale=0.20]{compound_rhythm.png}
\end{center}
Hier weist der nach oben gerichtete Balken auf ein zusammengesetzter Rhythmus hin, einen effektiven Akzent, und sollte nicht als Divisi interpretiert werden. \textbf{\circled{10} Wenn eine Passage für bestimmte Mitglieder desselben Orchestergruppe gilt}, wird die folgende Syntax verwendet: \textbf{,,1."} bedeutet, dass nur das erste Mitglied der Gruppe spielen soll. \textbf{,,1.|2."} bedeutet, dass nur das erste und zweite Mitglied der Gruppe spielen soll. \textbf{,,2.:"} gibt an, dass alle Mitglieder außer dem ersten spielen sollen.
\endgroup

\begingroup
\textbf{Rohrblattinstrumente: \circled{1} Mehrklänge} werden mit Griffdiagrammen oberhalb des Grundtons angezeigt. \\ \textbf{\circled{2} Rhythmisierte Klangfarbenänderungen} werden als eingekreiste Zahl über einer Note notiert ( z. B. \circled{1}, \circled{2} oder \circled{3} ), wobei höhere Zahlen eine größere Abweichung in Klangfarbe und Tonhöhe bedeuten.
\endgroup

\begingroup
\textbf{Fagotte: \circled{1} Pizzicato} wird mit gekreuzten Notenköpfen notiert, und \textbf{schmetternde Klänge} werden mit einem Pfeil auf dem Notenhals in Richtung des Notenkopfes notiert, wie in Allgemein unter Punkt 9 dargestellt. Die Anleitungen für diese Techniken sind so:
\begin{center}
,,Das \textbf{Pizzicato} entsteht beim Fagott durch sehr kurzes Schnalzen der Rohrspitze auf den Lippen . . . Notwendig ist nur eine kurze, trockene Lippenbewegung auf der Rohrspitze; auf den Blasdruck des Zwerchfells wird verzichtet. Die Bewegung gleicht der Aussprache des Buchstabens >>P<<." \\

- Pascal Gallois, \textit{Die Spieltechnik des Fagotts} Seite 47 Abs. 1
\end{center}
\endgroup

\begingroup
\begin{center}
,,Der \textbf{schmetternde Klang} ähnelt dem Pizzicato, ist jedoch immer sehr kräftig und erfordert den Blasdruck des Zwerchfells. Er entsteht durch ein sehr hartes und energisches Schnalzen der Rohrspitze auf den Lippen und erinnert an den schmetternden Klang einer Posaune . . ." \\

- Pascal Gallois, \textit{Die Spieltechnik des Fagotts} Seite 47 Abs. 3
\end{center}
\endgroup

\end{document}