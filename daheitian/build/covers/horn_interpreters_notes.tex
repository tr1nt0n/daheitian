\documentclass[12pt]{article}
\usepackage{fontspec}
\usepackage[utf8]{inputenc}
\setmainfont{Bodoni 72 Book}
\usepackage[paperwidth=9in,paperheight=12in,margin=1in,headheight=0.0in,footskip=0.5in,includehead,includefoot,portrait]{geometry}
\usepackage[absolute]{textpos}
\TPGrid[0.5in, 0.25in]{23}{24}
\parindent=0pt
\parskip=12pt
\usepackage{nopageno}
\usepackage{graphicx}
\graphicspath{ {./images/} }
\usepackage{amsmath}
\usepackage{tikz}
\newcommand*\circled[1]{\tikz[baseline=(char.base)]{
            \node[shape=circle,draw,inner sep=1pt] (char) {#1};}}

\begin{document}

\begingroup
\begin{center}
\huge HINWEISE FÜR DIE INTERPRETEN
\end{center}
\endgroup

\vspace*{2\baselineskip}

\begingroup
\textbf{Allgemein: \circled{1} Vorzeichen} werden für jeden Takt gesetzt, aber sie werden nochmal gesetzt, wenn die gleiche Note später im selben Takt auftritt - außer die Note wird unmittelbar wiederholt. \textbf{\circled{2} Dynamik}, gefolgt von einem Pluszeichen, bedeutet, dass zwischen der notierten Dynamik und der nächsten Standarddynamikstufe gespielt werden soll. So zeigt \textbf{pp +} an, dass zwischen Pianissimo und Piano gespielt werden soll. \textbf{\circled{3} Flache Glissandi} werden in ähnlicher Weise wie Bindebögen verwendet, aber während Bindebögen auf die Darstellung metrischer Pulsgruppierungen während einer einzelnen Note beschränkt sind, binden flache Glissandi komponierte Rhythmen, um als \textbf{Ankernoten für dynamische Veränderungen} innerhalb einer anhaltenden einzelnen Note verwendet zu werden. Die Interpreten müssen sich nicht darum kümmern, ob ein solches flaches Glissando ein ,,echtes Glissando" eines Halbtons ist, da ein solches ,,echtes Glissando" \textbf{immer auch mit Vorzeichen} angezeigt wird. \textbf{\circled{4} Instrumentaltechniken} gelten nur für die Note, mit der sie verbunden sind. Wenn eine Technik länger als eine Note bestehen muss, umspannt eine \textbf{Hakenlinie} die Musik, in der die Technik aktiv ist. \textbf{\circled{5} Pfeile} kennzeichnen einen allmählichen Wechsel von einer Technik oder einem Tempo zu einer anderen. \textbf{\circled{6} Vorschlagsnoten vor} einer Note sollten direkt vor dem Rhythmus gespielt werden, Vorschlagsnoten \textbf{nach} einer Note sollten ganz am Ende der Dauer der betreffenden Note gespielt werden. \textbf{\circled{7} Fermaten} und ihre Längen sind wie folgt zu interpretieren:
\endgroup

\begingroup
\hspace{67mm} \circled{1} Sehr kurz \hspace{5mm} \includegraphics[scale=0.05]{ganz_kurz.png}
\endgroup

\begingroup
\hspace{67mm} \circled{2} Kurz \hspace{13mm} \includegraphics[scale=0.045]{kurz.png}
\endgroup

\begingroup
\hspace{67mm} \circled{3} Mittel \hspace{10mm} \includegraphics[scale=0.04]{mittel.png}
\endgroup

\begingroup
\hspace{67mm} \circled{4} Lang \hspace{12mm} \includegraphics[scale=0.035]{lang.png}
\endgroup

\begingroup
\hspace{67mm} \circled{5} Sehr lang \hspace{4.5mm} \includegraphics[scale=0.045]{sehr_lang.png}
\endgroup

\begingroup
\textbf{\circled{8}} Da diese Parameter von Instrument zu Instrument und von Lautstärke zu Lautstärke variieren können, wird die \textbf{höchst- bzw. tiefstmögliche Tonhöhe} eines Instruments, die nicht auf eine bestimmte Harmonie, sondern auf einen \textbf{Effekt} abzielt, mit einem \textbf{nach oben bzw. nach unten gerichteten dreieckigen Notenkopf} angezeigt.
\endgroup

\pagebreak

\begingroup
\textbf{\circled{9}} Die in diesem Stück verwendeten \textbf{gleichschwebenden Intervalle} sind \textbf{Halbtöne, Vierteltöne} und \textbf{Achteltöne.} Ihre Symbole lauten wie folgt:
\endgroup

\begingroup
\hspace{67mm} \circled{1} Ein Viertelton höher \hspace{5mm} \includegraphics[scale=0.06]{qtone_sharp.png}
\endgroup

\begingroup
\hspace{67mm} \circled{2} Ein Viertelton tiefer \hspace{5mm} \includegraphics[scale=0.07]{qtone_flat.png}
\endgroup 

\begingroup
\circled{3} Jedes Vorzeichen kann mit einem Pfeil oben oder unten verändert werden, was bedeutet, dass die Tonhöhe um einen Achtelton erniedrigt erhöht oder erniedrigt wird. \\
\begin{center}
\includegraphics[scale=0.07]{eighthtone.png}
\end{center}
\endgroup

\begingroup
\textbf{\circled{10} Rationale Intervalle} werden durch die Verwendung des \textbf{Helmholtz-Ellis-Vorzeichensystems} in Kombination mit \textbf{Cent-Abweichungen von der gleichschwebenden Stimmung} für die Verwendung mit einem elektronischen Stimmgerät angegeben. Wenn keine Beispieltonhöhe mit der Cent-Abweichung angegeben ist, ist die Markierung eine Abweichung vom \textbf{nächstgelegenen ,,Standard" Vorzeiche.} In Ermangelung elektronischer Stimmgeräte sind Näherungswerte für diese Abweichungen zulässig. Wenn die Helmholtz-Ellis-Notation nicht angegeben ist, sind die Tonhöhen wie üblich zu spielen. \textbf{\circled{11}}  Im Allgemeinen bedeutet ein mehrstimmiges Notensystem ein traditionelles \textbf{Divisi}. \textbf{\circled{12} Wenn eine Passage für bestimmte Mitglieder desselben Orchestergruppe gilt}, wird die folgende Syntax verwendet: \textbf{,,1."} bedeutet, dass nur das erste Mitglied der Gruppe spielen soll. \textbf{,,1.|2."} bedeutet, dass nur das erste und zweite Mitglied der Gruppe spielen soll. \textbf{,,2.:"} gibt an, dass alle Mitglieder außer dem ersten spielen sollen.
\endgroup

\begingroup
\textbf{Hörner: \circled{1} Diese Partitur ist so transponiert}, dass die notierte Tonhöhe \textbf{eine Quinte} über der klingenden Tonhöhe liegt.
\endgroup

\end{document}