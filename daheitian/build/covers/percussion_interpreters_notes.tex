\documentclass[12pt]{article}
\usepackage{fontspec}
\usepackage[utf8]{inputenc}
\setmainfont{Bodoni 72 Book}
\usepackage[paperwidth=9in,paperheight=12in,margin=1in,headheight=0.0in,footskip=0.5in,includehead,includefoot,portrait]{geometry}
\usepackage[absolute]{textpos}
\TPGrid[0.5in, 0.25in]{23}{24}
\parindent=0pt
\parskip=12pt
\usepackage{nopageno}
\usepackage{graphicx}
\graphicspath{ {./images/} }
\usepackage{amsmath}
\usepackage{tikz}
\newcommand*\circled[1]{\tikz[baseline=(char.base)]{
            \node[shape=circle,draw,inner sep=1pt] (char) {#1};}}

\begin{document}

\begingroup
\begin{center}
\huge HINWEISE FÜR DIE INTERPRETEN
\end{center}
\endgroup

\vspace*{2\baselineskip}

\begingroup
\textbf{Allgemein: \circled{1} Vorzeichen} werden für jeden Takt gesetzt, aber sie werden nochmal gesetzt, wenn die gleiche Note später im selben Takt auftritt - außer die Note wird unmittelbar wiederholt. \textbf{\circled{2} Dynamik}, gefolgt von einem Pluszeichen, bedeutet, dass zwischen der notierten Dynamik und der nächsten Standarddynamikstufe gespielt werden soll. So zeigt \textbf{pp +} an, dass zwischen Pianissimo und Piano gespielt werden soll. \textbf{\circled{3} Flache Glissandi} werden in ähnlicher Weise wie Bindebögen verwendet, aber während Bindebögen auf die Darstellung metrischer Pulsgruppierungen während einer einzelnen Note beschränkt sind, binden flache Glissandi komponierte Rhythmen, um als \textbf{Ankernoten für dynamische Veränderungen} innerhalb einer anhaltenden einzelnen Note verwendet zu werden. Die Interpreten müssen sich nicht darum kümmern, ob ein solches flaches Glissando ein ,,echtes Glissando" eines Halbtons ist, da ein solches ,,echtes Glissando" \textbf{immer auch mit Vorzeichen} angezeigt wird. \textbf{\circled{4} Instrumentaltechniken} gelten nur für die Note, mit der sie verbunden sind. Wenn eine Technik länger als eine Note bestehen muss, umspannt eine \textbf{Hakenlinie} die Musik, in der die Technik aktiv ist. \textbf{\circled{5} Pfeile} kennzeichnen einen allmählichen Wechsel von einer Technik oder einem Tempo zu einer anderen. \textbf{\circled{6} Vorschlagsnoten vor} einer Note sollten direkt vor dem Rhythmus gespielt werden, Vorschlagsnoten \textbf{nach} einer Note sollten ganz am Ende der Dauer der betreffenden Note gespielt werden. \textbf{\circled{7} Fermaten} und ihre Längen sind wie folgt zu interpretieren:
\endgroup

\begingroup
\hspace{67mm} \circled{1} Sehr kurz \hspace{5mm} \includegraphics[scale=0.05]{ganz_kurz.png}
\endgroup

\begingroup
\hspace{67mm} \circled{2} Kurz \hspace{13mm} \includegraphics[scale=0.045]{kurz.png}
\endgroup

\begingroup
\hspace{67mm} \circled{3} Mittel \hspace{10mm} \includegraphics[scale=0.04]{mittel.png}
\endgroup

\begingroup
\hspace{67mm} \circled{4} Lang \hspace{12mm} \includegraphics[scale=0.035]{lang.png}
\endgroup

\begingroup
\hspace{67mm} \circled{5} Sehr lang \hspace{4.5mm} \includegraphics[scale=0.045]{sehr_lang.png}
\endgroup

\begingroup
\textbf{\circled{8}}  Im Allgemeinen bedeutet ein mehrstimmiges Notensystem ein traditionelles \textbf{Divisi}. Eine Ausnahme bilden Passagen wie Takt 150 in den tiefe Rohrblattinstrumente. \\
\begin{center}
\includegraphics[scale=0.20]{compound_rhythm.png}
\end{center}
Hier weist der nach oben gerichtete Balken auf ein zusammengesetzter Rhythmus hin, einen effektiven Akzent, und sollte nicht als Divisi interpretiert werden. 
\endgroup

\begingroup
\textbf{Schlagzeug: \circled{1} Die Instrumente} und ihre \textbf{Werkzeug} des \textbf{ersten} Schlagzeuger sind so: \\
a.) Ein kleiner ( hoher ) \textbf{Triangel,} und ein \textbf{Triangel Schlägel} \\
b.) Ein \textbf{Bangu \setmainfont{Source Han Serif SC Bold}\selectfont{ ( 板鼓 ), } }und zwei kleine \textbf{Bambusstäbchen} ( Diese können bei Bedarf durch \textbf{Trommelstöcke} ersetzt werden, obwohl dies nicht bevorzugt wird ) \\
c.) Ein Satz \textbf{Röhrenglocken,} und zwei \textbf{Röhrenglockenschlägel} \\
d.) Ein große \textbf{Tanggu \setmainfont{Source Han Serif SC Bold}\selectfont{ ( 堂鼓 ), } }und zwei \textbf{Trommelstöcke}\\
\textbf{\circled{2} Die Instrumente} und ihre \textbf{Werkzeug} des \textbf{zweiten} Schlagzeuger sind so: \\
a.) Ein \textbf{Glockenspiel}, und zwei kleine \textbf{Plastikschlägel} \\
b.) Ein kleiner \textbf{Amboss}, und zwei kleine \textbf{Hämmerchen} \\
c.) Ein kleiner \textbf{Gong der chinesische Oper\setmainfont{Source Han Serif SC Bold}\selectfont{ ( 小鑼 ) } }  \\
d.) Ein mittelgroßer \textbf{Gong der chinesische Oper \setmainfont{Source Han Serif SC Bold}\selectfont{ ( 中型鑼 ), } } und zwei kleine, harte \textbf{Gongschlägel}\\
e.) Ein großer \textbf{Tam-Tam} ( vorbereitet mit \textbf{Ketten} an der Vorderseite ), und ein \textbf{Bogen} \\
f.) Ein kleiner ( hoher ) \textbf{Triangel,} und ein \textbf{Triangel Schlägel} \\
\endgroup

\end{document}